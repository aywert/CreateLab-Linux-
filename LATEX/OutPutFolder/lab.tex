\documentclass[a4paper,12pt]{article} % тип документа
%  Русский язык
\usepackage{multirow}
\usepackage{wrapfig}
\usepackage[T2A]{fontenc}			% кодировка
\usepackage[utf8]{inputenc}			% кодировка исходного текста
\usepackage[english,russian]{babel}	% локализация и переносы
\usepackage{graphicx}
\usepackage{todonotes}
% Математика
\usepackage{amsmath,amsfonts,amssymb,amsthm,mathtools}
\usepackage{hyperref}
% графики
\usepackage{pgfplots}
\pgfplotsset{compat=1.9}
\begin{document}
\begin{center}
{\large МОСКОВСКИЙ ФИЗИКО-ТЕХНИЧЕСКИЙ ИНСТИТУТ (НАЦИОНАЛЬНЫЙ ИССЛЕДОВАТЕЛЬСКИЙ УНИВЕРСИТЕТ)}
\end{center}
\begin{center}
{\largeФизтех-школа Радиотехники и Компьютерных Технологий}
\end{center}
\vspace{3.5cm}
\vspace{0.1cm}
{\huge
\begin{center}
{\bf Дифференциальная работа 3.3.3}\
\end{center}
}
\vspace{5cm}
{\LARGE Авторы:\\ Мовсесян Михаил \\
\newline
Б01-403}
\end{flushright}
\vspace{1.5cm}
\begin{center}
Долгопрудный 2024
\end{center}
\end{titlepage}
\begin{table}[h!]
   \begin{center}
   \caption{THE best table}
\end{document}
